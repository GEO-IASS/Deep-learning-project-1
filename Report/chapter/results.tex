\chapter{Results}
\label{chp:res}

This chapter describes the result of the implemented convolutional neural network applied to the CIFAR-10 dataset. It further presents the results of adjusting the hyperparameters: Learning rate, L2 Regularization, Dropout and Filter size.

First a coarse grain search is done followed by a fine grain search

\section{Coarse grain  search}
The coarse grain search is applied to identity a reasonable range for the different hyperparameters.

\subsection{Learning rate}
Based on the default learning rate of ADAM being 0.001 it was chosen to do 3 runs and investigate the loss function. The three learning rates were: 0.01, 0.001, 0.0001. The three graphs are shown in \ref{fig:lr_loss}. It can be seen that the learning rate of 0.01 is too high, while a learning rate of 0.0001 is too low. The learning rate of 0.001 was the best fit for the model of the three evaluated learning rates. Learning rates close to this value are investigated in the fine grain search.
	
\myFigure{learningrate/loss_overview.png}{Loss function for a learning rate of 0.01, 0.001, 0.0001. Visualization from Tensorboard with some smoothing applied.}{fig:lr_loss}{1}

\FloatBarrier
\subsection{L2 Regularization}
The following evaluation of L2 Regularization will be using the learning rate of 0.001. First two different L2 regularization penalties were tested: 0.01, 0.001.  The training and validation accuracy were plottet for both as illustrated in figure \ref{fig:l2pen_1_train} and \ref{fig:l2pen_1_valid}. A penalty of 0.01 showed little overfitting and both converged around 65\% but the accuracy was quite low which implies that the regularization term was overwhelming the data loss. For 0.01 the result shows quite a difference between training and validation accuracy, which implies overfitting for the training data. 

\myFigure{l2penalty/trainAcc_1}{Training accuracy with a regularization penalty of 0.01 and 0.001. Visualization from Tensorboard with some smoothing applied.}{fig:l2pen_1_train}{1}

\myFigure{l2penalty/validAcc_1}{Validation accuracy with a regularization penalty of 0.01 and 0.001. Visualization from Tensorboard with some smoothing applied.}{fig:l2pen_1_valid}{1}

A value between these two  would therefore be preferable. An L2 Penalty of 0.005 was tested with model. This is illustrated in figure \ref{fig:l2pen_2_train} and \ref{fig:l2pen_2_valid}. This gave less overfitting but a smaller validation accuracy than with 0.001. Based on the results from 0.005 the L2 penalty was lowered to 0.0025. This is also shown in figure \ref{fig:l2pen_2_train} and \ref{fig:l2pen_2_valid}. An L2 penalty of 0.0025 gave very little overfitting and approximately the same validation accuracy as 0.001. The graph might also indicate that if more epochs were being execuated a penalty of 0.0025 would yield the highest validation accuracy.

\myFigure{l2penalty/trainAcc_2}{Training accuracy with a regularization penalty of 0.0025 and 0.005. Visualization from Tensorboard with some smoothing applied.}{fig:l2pen_2_train}{1}

\myFigure{l2penalty/validAcc_2}{Validation accuracy with a regularization penalty of 0.0025 and 0.005. Visualization from Tensorboard with some smoothing applied.}{fig:l2pen_2_valid}{1}

\FloatBarrier
\subsection{Dropouts}
To further achieve less overfitting dropout can also be applied. In general it is good to combine L2 and dropout. A keep probability of 0.5 is a reasonable default, therefore in the previous executions this dropout rate was used. The following keep probabilities of 0.25, 0.5, 0.75 and 1 were also evaluated. A probability of 0.25 and 1 resulted in a lower validation accuracy. A probability of 0.25 still ensured a low overfitting while a probability of 1 worsened the overfitting. A probability of 0.5 and 0.75 both performed better with a little higher validation accuracy and small overfitting. In figure \ref{fig:drop_1_train} and \ref{fig:drop_1_valid} the training and validation accuracy is illustrated.

\myFigure{drop/train_acc.png}{Training accuracy with a dropout keep probabilities of 0.25, 0.5, 0.75 and 1. Visualization from Tensorboard with some smoothing applied.}{fig:drop_1_train}{1}

\myFigure{drop/valid_acc.png}{Validation accuracy with a dropout keep probabilities of 0.25, 0.5, 0.75. Visualization from Tensorboard with some smoothing applied.}{fig:drop_1_valid}{1}

\FloatBarrier
\subsection{Convolutional layer filter size}
Until now the convolutional layers have both used a filter size of 5x5. It will now be investigated whether a 3x3 filter might be a better choice for the model. In figure \ref{fig:filter_valid} the validation accuracy is illustrated when using filters of size 3x3 and 5x5. A filter size of 5x5 gave a little better result.

\myFigure{filter/valid_acc.png}{Validation accuracy with 3x3 and 5x5 filters. Visualization from Tensorboard with some smoothing applied.}{fig:filter_valid}{1}


\section{Fine grain search}
After finding coarse values for the hyperparameters a more fine grained search was applied. This was done with a random search where all three hyperparameters were chosen randomly in some range and then run.

First a run with the parameters which were found to be most optimal in the coarse grained part. The run is instead done with 60 epochs to see if a higher validation accuracy can be achieved. In general more epochs gave a better validation accuracy. During the random search many networks perfoming similar were found as seen below. 

\section{A deeper convolutional network}
After evaluating hyperparameters for the convolutional network the deeper network was simply tested with the same parameters. As shown in figure bla \todo{insert figure ref}  this gave a higher validation accuracy.


