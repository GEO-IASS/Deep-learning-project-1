\chapter{Conclusion}
\label{chp:conc}

\section{Convolutional network layers}

Traditional neural network with fully-connected layers does not perform well with images classification. This is because they create the number of weights due to the size of the image. For instance if an input image is 800x600x3 (wide, height, depth) the number of weights would be $800 * 600 * 3 = 1,440,000$. Due to the large number of parameters, it is easy to se that this is wasteful technique to work with images and furthermore it will lead to an overfitting of the input images. On the other hand a
Convolutional nerual network also called \emph{CNN} would be a better solution when working on images classification because it assumes that the input is an image or on the same form. The power in CNN is that the layers are three dimensional and the neurons can share weights and bias values. This reduces the amount of parameters significantly. The CNN consist of three different layers, namely a \emph{convolutional layer}, a \emph{pooling layer} and a \emph{fully-connected layer}. These three layers will be described in the following sections.
 
\subsection{Convolutional layer}
This is the layer where most is happening. The convolutional layer creates a number of filters with a height, width and a depth. Normally values for the height and width is three or five. The depth should be the same as the input image. So a dimension for filter for a 32x32x3 input image could be 5x5x3. Each filter looks at different things in the image eg. oriented edges, blotchs or colors. Each filter is convolve across the input image along the height and width. To specify how it has to convolve three hyperparameters need to be set namely \emph{depth}, \emph{stride} and \emph{zero-padding}. Depth is the number of neurons that looks at the same filter. Stride is the number of pixels the filter should convolve with across the image. Finally zero-padding is used to determine the size of the output volume. This is often the same as the input image. 

\subsection{Pooling layer}

\subsection{Fully-connected layer}